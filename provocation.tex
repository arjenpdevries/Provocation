\documentclass[a4paper,twoside,12pt]{article}
\usepackage{graphicx,fancyhdr, url}
\usepackage[utf8]{inputenc}
\usepackage[english]{babel}
%\usepackage{lmodern}

\title{Citizens in Data Land}

\author{Arjen P. de Vries\\Radboud Universiteit\\arjen@acm.org}
\date{}

\setlength{\headheight}{28pt}
\pagestyle{fancy}
\fancyhf{}
\fancyhead[R]{Citizens in Data Land}
\fancyhead[L]{Arjen P. de Vries}
\fancyfoot[C]{\thepage}

\setlength{\parindent}{0em}
\setlength{\parskip}{1em}

\begin{document}
\maketitle

\begin{abstract}
Provocation for the workshop `10 Years of Profiling the European Citizen', June 12--13, 2018, Brussels, for the panel on `Legal and political theory in data driven environments'.
\end{abstract}

The informed reader has immediately recognised the use of \LaTeX{} and infers correctly that this \emph{provocation} is written by a computer scientist. The author is indeed trained as computer scientist, and the first thing he did upon receiving the invitation to join this event was to look-up the meaning of the term `provocation' using an internet search engine; I might as well admit that, given that I shared this bit of information about myself already with one of the largest tech companies in the world. The title of the panel I was asked to contribute to revealed more gaps in my background knowledge -- for me, the immediate association with `political theory' is the title of a Coldplay song. Wikipedia to the rescue, even if I would tell my students to not simply rely on the information in the online encyclopedia when it concerns my area of expertise... At this point in my contribution, you know half of what you also would have learned by looking at one of the social media sites where I have an account.\footnote{%
Computer scientist and entrepreneur. Information access \& integration of IR and DB. And Indie music.}

Since we celebrate the $10^\textit{th}$ anniversary of Profiling the European Citizen, I do not need to introduce the case of Facebook, Cambridge Analytica and the likely abuse of personal information to influence democratic elections. The reasons for introducing European data protection legislation could not have had a better advocate than that scandal; and I consider ourselves very lucky with that media storm, given the notable nuisance of a flood of privacy related notifications that filled our email boxes this month.\footnote{%
Perhaps the most annoying response to the GDPR I have seen were the US-based newspapers that opted to exclude web site visitors with European IP addresses from reading their articles.}
Of course, the simple fact that you can find out easily personal information via a web search using my name (you need to include the middle initial, as last name `De Vries' is not a very discriminating feature) is no issue of concern -- this information is a public self-description I contributed voluntarily to the online world, as a citizen of \emph{Data Land}, and self-describes me to advertise who I am and why you might want to connect to me.

What does (and should) concern us, is the information about us that we give away implicitly, through usage of an online service, without us being aware. Quoting Hildebrandt and Gutwirth from the closing chapter of the book that brings us to Brussels today:
\begin{quote}
For individual citizens to regain some kind of control over the way they live their lives, access is
needed to the profiles applied to them. This will require both legal (rights to transparency)
and technological tools (the means to exercise such rights).
\end{quote}

I will take the position that we have been quite successful in creating a legal framework that gives people the power to claim substantial rights to their data, even if we have not yet gained a lot of experience with the law being tested on its practical usefulness yet. However, we certainly cannot sit back and wait, and expect the GDPR to save our privacy from organisations' hunger for data. Where we really lack progress, is the tool support. Wouldn't it be so much easier to exercise our right on data portability if only we knew what to port that data to? 

As it is, the balance of power between individual citizens and the organisations they deal with, both public and private, shifts back only a tiny fraction of how it could have shifted; if only citizens had the means to take control of their data, including the traces they leave online.

Here, I strongly believe that the answer can only be found in decentralisation. We create the data that we are concerned with on our own devices. However, we have been seduced \emph{en masse} to give up control of this data by the convenience of having services managed in the cloud, and we happily give it all away without complete consideration of the consequences. Sure, using our re-gained rights, we could try to claim this data back, or exercise control over the way it is used -- but would it not be so much easier to simply keep it for ourselves in the first place?!

Considering the specific case of claiming back control over our online social media data, I would like to draw your attention to a new W3C standard called ActivityPub\footnote{%
\url{https://www.w3.org/TR/activitypub/}}.
The standard has been granted the status of `recommendation' since January $23^\textit{rd}$, 2018, and is implemented in a number of open source projects, including a very interesting one called Mastodon: that facilitates the communication between over 1 million registered users across thousands of Mastodon instances. Other projects are creating alternatives for instagram and youtube, and taken all together, these services that exchange social information using ActivityPub (and its predecessors) form what is now called the Fediverse (a partial blend of federated and universe).

In the case of Mastodon, an instance can host an online community that clusters together for many reasons - 

an example is a recent one created for `all people with an email address from University of Twente'; MIT runs another educational instance in the USA; and I myself created an instance for the Information Retrieval community. 





%The term appeared in the 5th century BC, to denote the political systems then existing in Greek city-states, notably Athens, to mean "rule of the people", in contrast to aristocracy(ἀριστοκρατία, aristokratía), meaning "rule of an elite". While theoretically these definitions are in opposition, in practice the distinction has been blurred historically.[5] The political system of Classical Athens, for example, granted democratic citizenship to free men and excluded slaves and women from political participation. In virtually all democratic governments throughout ancient and modern history, democratic citizenship consisted of an elite class until full enfranchisement was won for all adult citizens in most modern democracies through the suffrage movements of the 19th and 20th centuries.



\bibliographystyle{alpha}
\bibliography{provocation.bib}

\end{document}

\url{http://www.samkinsley.com/2014/06/23/culture-technology-the-text-of-my-provocation/}
\url{https://www.cambridge.org/core/journals/legal-theory/article/in-defense-of-contentindependence/9C9463577AD7AFCFE0A335998EC12571}
