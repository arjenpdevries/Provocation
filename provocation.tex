\documentclass[a4paper,twoside,12pt]{article}
\usepackage{graphicx,fancyhdr, url}
\usepackage[utf8]{inputenc}
\usepackage[english]{babel}
\usepackage{lmodern}

\title{Citizens in Data Land}

\author{Arjen P. de Vries\\Radboud Universiteit\\arjen@acm.org}
\date{}

\setlength{\headheight}{28pt}
\pagestyle{fancy}
\fancyhf{}
\fancyhead[R]{Citizens in Data Land}
\fancyhead[L]{Arjen P. de Vries}
\fancyfoot[C]{\thepage}

\setlength{\parindent}{0em}
\setlength{\parskip}{1em}

\begin{document}
\maketitle

\begin{abstract}
Provocation for the workshop `10 Years of Profiling the European Citizen', June 12--13, 2018, Brussels, for the panel on `Legal and political theory in data driven environments'.
\end{abstract}

The informed reader immediately recognises the use of \Latex and draws the conclusion that this provocation is written by a computer scientist. The author is only trained as computer scientist, and the first thing he had to do after receiving the invitation to join this event was to look-up the meaning of `provocation' using an internet search engine; I might as well admit that to you, given that I shared this bit of information about my profile already with one of the largest tech companies in the world.



2018 marks the 10 th anniversary of Profiling the European Citizen (ed. Mireille Hildebrandt and Serge
Gutwirth), which brought together lawyers, computer scientists and philosophers around emerging
practices of data mining, knowledge discovery in data bases (KDD) and their application in a variety
of domains.
The title of the book turns out to have been prophetic, touching upon a series of implications of what
has now been coined as the micro targeting of individuals as consumers and citizens, based on
machine learning and AB testing.
To celebrate – or even to mourn - the relevance of the volume, LSTS is organizing a seminar to reflect
on current affairs and further implications.



Grappig:
\url{https://en.m.wikipedia.org/wiki/Provocation_(legal)}


\url{http://www.samkinsley.com/2014/06/23/culture-technology-the-text-of-my-provocation/}


Political Theory track provides students with the conceptual tool-kits to analyse the complex normative and methodological issues of contemporary political life

\url{https://www.cambridge.org/core/journals/legal-theory/article/in-defense-of-contentindependence/9C9463577AD7AFCFE0A335998EC12571}

The term appeared in the 5th century BC, to denote the political systems then existing in Greek city-states, notably Athens, to mean "rule of the people", in contrast to aristocracy(ἀριστοκρατία, aristokratía), meaning "rule of an elite". While theoretically these definitions are in opposition, in practice the distinction has been blurred historically.[5] The political system of Classical Athens, for example, granted democratic citizenship to free men and excluded slaves and women from political participation. In virtually all democratic governments throughout ancient and modern history, democratic citizenship consisted of an elite class until full enfranchisement was won for all adult citizens in most modern democracies through the suffrage movements of the 19th and 20th centuries.


\newpage







Myself a researcher with a background in data management and information retrieval, I have long been intrigued by the idea that data powers insight to help improve science and society. I recall being excited by the wonderful bundle of essays titled `The Fourth Paradigm: Data-intensive Scientific Discovery',%
\footnote{\url{%
https://web.archive.org/web/20091223044640/http://research.microsoft.com/en-us/collaboration/fourthparadigm/4th_paradigm_book_complete_lr.pdf%
}}
edited by Microsoft Research, that showcases a kaleidoscope of scientific progress enabled by the use of computers to gain understanding from data created and stored electronically.
But, the impact of data science reaches far beyond science itself. Can you name one organisation, public or private, that is not looking to hire data scientists? 


\bibliographystyle{alpha}
\bibliography{provocation.bib}

\end{document}