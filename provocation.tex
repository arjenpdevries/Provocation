\documentclass[a4paper,twoside,12pt]{article}
\usepackage{graphicx,fancyhdr, url}
\usepackage[utf8]{inputenc}
\usepackage[english]{babel}
%\usepackage{lmodern}

\title{Citizens in Data Land}

\author{Arjen P. de Vries\\Radboud Universiteit\\arjen@acm.org}
\date{}

\setlength{\headheight}{28pt}
\pagestyle{fancy}
\fancyhf{}
\fancyhead[R]{Citizens in Data Land}
\fancyhead[L]{Arjen P. de Vries}
\fancyfoot[C]{\thepage}

\setlength{\parindent}{0em}
\setlength{\parskip}{1em}

\begin{document}
\maketitle

\begin{abstract}
Provocation for the workshop `10 Years of Profiling the European Citizen', June 12--13, 2018, Brussels, for the panel on `Legal and political theory in data driven environments'.
\end{abstract}

\section*{Opening Statement}
I would like to start with a quote of Hildebrandt and Gutwirth, drawn from the closing chapter of the book that brings us to Brussels today:
\begin{quote}
For individual citizens to regain some kind of control over the way they live their lives, access is
needed to the profiles applied to them. This will require both legal (rights to transparency)
and technological tools (the means to exercise such rights).
\end{quote}

I will take the position that, from these two requirements, we have been quite successful in creating a legal framework that gives people the power to claim substantial rights over their personal data; even if we have not yet gained much experience with the law being tested on its practical usefulness yet, restrictions have been imposed upon those parties processing personal data, including the data minimisation and data portability requirements. Where we really lack progress, is the tool support. Wouldn't it be so much easier to exercise our right on data portability if only we knew what to port that data to? 

\section*{Profiling}
The informed reader has immediately recognised the use of \LaTeX{} and infers correctly that this \emph{provocation} is written by a computer scientist. The author is indeed trained as computer scientist, and the first thing he did upon receiving the invitation to join this event was to look-up the meaning of the term `provocation' using an internet search engine; I might as well admit that, given that I shared this bit of information about myself already with one of the largest tech companies in the world. The title of the panel I was asked to contribute to revealed more gaps in my background knowledge -- for me, the immediate association with `political theory' is the title of a Coldplay song. Wikipedia to the rescue, even if I would tell my students to not simply rely on the information in the online encyclopedia when it concerns my area of expertise... At this point in my contribution, you know half of what you also would have learned by looking at one of the social media sites where I have an account.\footnote{%
Computer scientist and entrepreneur. Information access \& integration of IR and DB. And Indie music.}

Since we celebrate the $10^\textit{th}$ anniversary of Profiling the European Citizen, I do not need to introduce the case of Facebook, Cambridge Analytica and the likely abuse of personal information to influence democratic elections. The reasons for introducing European data protection legislation could not have had a better advocate than that scandal; and I consider ourselves very lucky with that media storm, given the notable nuisance of a flood of privacy related notifications that filled our email boxes this month.\footnote{%
Perhaps the most annoying response to the GDPR I have seen were the US-based newspapers that opted to exclude web site visitors with European IP addresses from reading their articles.}
Of course, the simple fact that you can find out easily personal information via a web search using my name (you need to include the middle initial, as last name `De Vries' is not a very discriminating feature) is no issue of concern -- this information is a public self-description I contributed voluntarily to the online world, as a citizen of \emph{Data Land} to advertise to others who I am and why you might want to connect to me. 

What does (and should) raise objections, is the usage of information about us that we give away implicitly, through usage of an online service, without us being aware. It is not so easy to escape such profiling - a recent analysis of the CommonCrawl 2012 corpus found that the majority of sites contain trackers, even if websites with highly privacy-critical content are less likely to do so (60\% vs 90\% for other websites) \cite{schelter18}. This weekend, a chance conversation with an independent blogger taught me that their commissioning parties \emph{demand} Google Analytics based statistics: to generate any income as an online writer, sharing visit data from your blog with Google has become a \emph{de facto} prerequisite, even if you keep your site free from advertisements. 

\section*{Citizenship in Data Land}

Will the new legal rights (transparency and control) help enforce a new balance? We definitely cannot sit back and wait and expect the GDPR to save our privacy from organisations' hunger for data. If only citizens had the means to take control of their data, including the traces they leave online! I think that we have achieved very little progress with regard to the technological tools necessary to exercise our new rights. The current situation is that `we the people' give those who run the online services a \emph{carte blanche} to collect our data, legal framework in place only to make this collection more transparent (we hope). We lack the awareness that it is us, that it is -- to a large extent -- our own personal choice (mistake) that we turn ourselves (our data) into slavery of a few very large, omni-present organisations, that harvest personal data \emph{en masse}. If we do not modify our own online behaviour, the balance of power (between individual citizens and the organisations they deal with, both public and private) does indeed shift back to the citizen thanks to the GDPR; but only a tiny fraction of how it could shift.

We create our personal data on our own devices. However, we have been seduced to give up control of this data, by free will, in exchange for convenience: the convenience of having services managed in the cloud for us. We happily give our data away, without sufficient consideration of the consequences. Sure, using our re-gained rights, we could try to claim this data back, or exercise control over the way it is used -- but would it not be so much easier to simply keep it for ourselves in the first place?! 

I put my cards on two developments to help establish a renewed, better balance, where the people who create the data will be able to exercise a significantly larger degree of ownership over their data.

The first idea is to build systems that support our online information interactions differently, based on a principle to `simply' keep data where it originates: in your own device. As a proof of concept, we have developed a personal web archive and search system (WASP)\footnote{%
\url{https://github.com/webis-de/wasp/}},
that archives and indexes all your interactions with the Web, enabling effective re-finding of anything you have seen online (where those searches remain completely local). A more radical version of this idea would crawl and store a significant fraction of the Web local to your device, instead of in your favourite search engine's data centres.

The second principle would be to decentralise online services (or, better, emph{re-}decentralise the Web), especially promising to claim back control over online social media. ActivityPub\footnote{%
\url{https://www.w3.org/TR/activitypub/}}.
is a new W3C standard that has been granted the status of `recommendation' (since January $23^\textit{rd}$, 2018) and is implemented in an increasing number of open source projects, including a `decentralised version of Twitter' called Mastodon. ActivityPub facilitates the communication among thousands of Mastodon instances that together host over 1 million registered users. Other community projects create decentralised alternatives for Instagram (PixelFed), YouTube (PeerTube), and Medium (Plume); taken all together, this federated cooperation of decentralised online services that exchange social information using ActivityPub (and its predecessors) form what is now called \emph{the Fediverse} (a partial blend of federated and universe). 

Members of the Fediverse interact freely with each other, even if their accounts reside on different so-called `instances'. This enables communities to organise themselves, independent from large corporations that would like to collect this data in a huge centralised database. Examples of instances that serve a community include the recent Mastodon instance created for `all people with an email address from University of Twente', the MIT instance in the USA, and, an instance I created myself, aiming to be a new online home for the Information Retrieval community\footnote{%
Visit \url{https://idf.social/} or \url{https://mastodon.utwente.nl/} for more information.}.

\section*{Closing Statement}

The directions in which I seek a solution for better technological support are still a long way away of empowering the Citizens of Data Land. A big hurdle to take is how we get these novel solutions in a state that Data Land falls under `the rule of the people'. In other words, managing your personal data is a `21st century skill' that the Citizens in Data Land will have to master. If we do not pay attention, we may end up replacing the one `aristocracy', of an elite of large tech corporations, by a different `elite', one consisting of just those tech savvy people who know how to operate their own data infrastructure but exclude others from exercising the same level of control over their data. Paraphrasing Hildebrandt and Gutwirth instead of quoting them, one might say that `citizenship, participation in the creation of the common good and personal freedom cannot be taken for granted, they presume that citizens' \emph{acquire the competences to exercise control over what is known about them and by whom}.

\bibliographystyle{alpha}
\bibliography{provocation.bib}

\end{document}

\url{http://www.samkinsley.com/2014/06/23/culture-technology-the-text-of-my-provocation/}
\url{https://www.cambridge.org/core/journals/legal-theory/article/in-defense-of-contentindependence/9C9463577AD7AFCFE0A335998EC12571}
